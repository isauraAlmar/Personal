\PassOptionsToPackage{unicode=true}{hyperref} % options for packages loaded elsewhere
\PassOptionsToPackage{hyphens}{url}
%
\documentclass[]{article}
\usepackage{lmodern}
\usepackage{amssymb,amsmath}
\usepackage{ifxetex,ifluatex}
\usepackage{fixltx2e} % provides \textsubscript
\ifnum 0\ifxetex 1\fi\ifluatex 1\fi=0 % if pdftex
  \usepackage[T1]{fontenc}
  \usepackage[utf8]{inputenc}
  \usepackage{textcomp} % provides euro and other symbols
\else % if luatex or xelatex
  \usepackage{unicode-math}
  \defaultfontfeatures{Ligatures=TeX,Scale=MatchLowercase}
\fi
% use upquote if available, for straight quotes in verbatim environments
\IfFileExists{upquote.sty}{\usepackage{upquote}}{}
% use microtype if available
\IfFileExists{microtype.sty}{%
\usepackage[]{microtype}
\UseMicrotypeSet[protrusion]{basicmath} % disable protrusion for tt fonts
}{}
\IfFileExists{parskip.sty}{%
\usepackage{parskip}
}{% else
\setlength{\parindent}{0pt}
\setlength{\parskip}{6pt plus 2pt minus 1pt}
}
\usepackage{hyperref}
\hypersetup{
            pdfborder={0 0 0},
            breaklinks=true}
\urlstyle{same}  % don't use monospace font for urls
\usepackage{longtable,booktabs}
% Fix footnotes in tables (requires footnote package)
\IfFileExists{footnote.sty}{\usepackage{footnote}\makesavenoteenv{longtable}}{}
\usepackage{graphicx,grffile}
\makeatletter
\def\maxwidth{\ifdim\Gin@nat@width>\linewidth\linewidth\else\Gin@nat@width\fi}
\def\maxheight{\ifdim\Gin@nat@height>\textheight\textheight\else\Gin@nat@height\fi}
\makeatother
% Scale images if necessary, so that they will not overflow the page
% margins by default, and it is still possible to overwrite the defaults
% using explicit options in \includegraphics[width, height, ...]{}
\setkeys{Gin}{width=\maxwidth,height=\maxheight,keepaspectratio}
\setlength{\emergencystretch}{3em}  % prevent overfull lines
\providecommand{\tightlist}{%
  \setlength{\itemsep}{0pt}\setlength{\parskip}{0pt}}
\setcounter{secnumdepth}{0}
% Redefines (sub)paragraphs to behave more like sections
\ifx\paragraph\undefined\else
\let\oldparagraph\paragraph
\renewcommand{\paragraph}[1]{\oldparagraph{#1}\mbox{}}
\fi
\ifx\subparagraph\undefined\else
\let\oldsubparagraph\subparagraph
\renewcommand{\subparagraph}[1]{\oldsubparagraph{#1}\mbox{}}
\fi

% set default figure placement to htbp
\makeatletter
\def\fps@figure{htbp}
\makeatother


\date{}

\begin{document}

\hypertarget{final-project}{%
\section{Final Project}\label{final-project}}

\hypertarget{introduction}{%
\subsection{INTRODUCTION}\label{introduction}}

The goal of this chapter is to provide context about the environment
where this project has been developed. The first subsection will provide
a general overview of the project, explaining the antecedents and the
original idea.

As the project has been developed in collaboration with a company, the
second subsection will include some background information about the
company, Everis

The next part dives deeper into the structure of Everis, and will focus
on NextGen, Everis innovation area. The fourth subsection will explain
the structure of the Robotics/Augmented Workforce team, where I am
currently working. We will finish this part explaining the current needs
of the different projects in which my team is working, and how this
project is able to help.

\hypertarget{project-overview}{%
\subsubsection{Project overview}\label{project-overview}}

Robots are becoming increasingly present in everyday life, and are an
active area of research, where the theoretical applications are almost
infinite. However, the most interesting and powerful features, usually
involve processing a real-time stream audio or video stream, in search
of different patterns: such as words, objects, faces, movements\ldots{}
and then use machine-learning to generate a customised response, while
having several other threads running. This requires a high computing
power, and processing and storing big quantities of data.

As most of them do not have a significant computing power, this is a
challenge that must be solved in order to improve the capabilities of
current robots. There are several reasons for this hardware limitation.
Sometimes, a better chip simply doesn't fit into the available physical
space on the robot without changing the design, other times a customised
chip would be needed, increasing vastly the costs of mass-production of
the robots, and also sometimes it is caused by robot designers, who did
not think on that possible use of the robot. There are, of course, some
high-end robots whose hardware is not that constraining, but their
prices are usually very high, so few consumers have access to them.

That is why in my team came up with the idea of creating a server for
the robot to connect, and send information about the state of the
sensors and actuators. The server doesn't have the same hardware
constrains, so it could compute them and send the processed data to the
robot, so its role would only be performing the computed response.
Furthermore, if instead of only a server, you use ``the cloud'', you can
have as much computing power and storage as needed, with the only
requirement of having internet access. This would also open the door to
have different robots connected to the cloud, sharing a common
``knowledge base'', and allowing application developers to invent brand
new ways of exploiting this common information.

The combination of different trending technologies such as big data,
cloud computing, machine-learning and robotics can lead to a significant
advance in the field. Let's imagine that a girl named Lisa, enters into
a department store for the first time. At the entrance, she meets a
robot, who learns her name and her face, and with the help of the
wireless cameras and NFC sensors, analyses the behaviour of Lisa,
learning about what she purchased and what she didn't. If any other day
Lisa decides to go to another store of the same brand, their local robot
already knows who she is, her interests last time, and a record of what
she purchased in the past, so he can greet her ``Hey Lisa, do you
remember the jacket you tried on your last visit? Today you can buy it
with a -40\% discount!''.

So this is where this project comes. The main goal is to generate a
simple, yet scalable, proof of concept system which should allow robots
to send data to process and receive a processed output, and allowing a
second robot to access the knowledge generated by the first one. On the
educational side, this project will allow me to investigate different
technologies, and design the architecture of a complex cloud-based
system, while allowing different robots to interact with it.

\hypertarget{about-the-company}{%
\subsubsection{About the company}\label{about-the-company}}

Everis, an NTT Data Company, is a multinational consulting firm that
offers business and strategic solutions, development and maintenance of
technological applications and outsourcing services. The company has
grown organically, in double digits and reported 1.03 billion Euro in
turnover last year. It currently employs more than 20,000 professionals
at its offices and high-performance centres, located in 16 countries.
Its main operation areas are Western Europe and Latin America.

The company was originally named DMR Consulting and was founded in Spain
on October 1996, when a group of 5 entrepreneurs left their managerial
positions at a leading consulting company to create a new concept of
consulting. The idea was simple to explain, but complex to implement:
building from scratch a unique company model designed to attract and
retain the best talent, while fostering good people and collective
values. Our talent management model, the main asset of Everis, is taught
as a case study in leading business schools, as it excels in smartly
feeding both the brain (skills and professional growth) and the heart
(attitudes and values) of our employees. The result is summarised in our
claim: attitude makes the difference.

What Everis can do for companies:

\begin{itemize}
\item
  Disruption, why go far if you can go further
\item
  Consulting, successful Digital projects with Everis
\item
  Transformation we define pragmatic and realistic initiatives, making
  transformation happen.
\item
  Technology, we provide the best technology and enterprise solutions to
  create new experiences.
\item
  Operations, Delivering quality and efficient Managed Services
\end{itemize}

\hypertarget{about-the-department}{%
\subsubsection{About the department}\label{about-the-department}}

\begin{figure}
\centering
\includegraphics{./tex2pdf.7640/3edbaf91ead157a8954e82891b116d0225798070.shtml}
\caption{alt text}
\end{figure}

\begin{itemize}
\item
  Everis NEXT: A Big Data and Artificial Intelligence based system that
  gathers the global information of disruptive startups, scouts
  technologies and extracts market insights as well as the investment
  strategies of main multinational companies.
\item
  Augmented workforce: Artificial intelligence \& Robotics: The main
  goal of the team at Everis is to integrate the robots into the society
  to generate a global benefit. Instead of focusing on the robot, the
  main goal of the company is the relation between human-robot in order
  to generate a smooth and friendly interaction.
\item
  CLOQQ: CLOQQ (acronym for ``Crea Lo Que Quieras'' in English ``Create
  What You Want'') offers both face-to-face activities (workshops,
  events, etc.) and a complete online and content environment for girls
  and boys to create their own video games, apps, robots, animations and
  much more. Watch video.
\end{itemize}

https://www.Everis.com/spain/es/news/newsroom/Everis-lanza-cloqq-una-iniciativa-para-que-ninos-y-ninas-se-diviertan-aprendiendo-0
(12/11/2017)

\begin{itemize}
\tightlist
\item
  Everis INFINITE: Everis Infinite is a free software application from
  the Office Suites \& Tools subcategory, part of the Business category.
  The app is currently available in English and it was last updated on
  2017-01-20. The program can be installed on iOS.
  https://Everis-infinite-ios.soft112.com/ (12/11/2017)
\end{itemize}

\hypertarget{about-the-team}{%
\subsubsection{About the team}\label{about-the-team}}

(EXPLAIN MORE ABOUT THE PROJECTS THAT WE HAVE) Artificial intelligence
\& Robotics: The main goal of the team at Everis is to integrate the
robots into the society to generate a global benefit. Instead of
focusing on the robot, the main goal of the company is the relation
between human-robot in order to generate a smooth and friendly
interaction.

We have two areas of actuation: Business (non-industrial) where through
introducing robots in companies the workers will feel more comfortable.
Social focusing on using the robotics to improve the society. For
example, people with disabilities, improve learning or robot partner.

Robots that the team works with:

Pepper

Pepper is a robot from Aldebaran, a partner of Soft Bank Group. It is a
humanoid robot that is 1,20 meters tall, and it is created to interact
with it in different ways like having a vocal interaction, through its
tablet, it has different sensors, etc.

Jibo

Tiago

Nao

Pleo and Casper project

Sota

\hypertarget{what-they-have}{%
\subsubsection{What they have}\label{what-they-have}}

Before I start the project, the company had a small Proof of Concept
(PoC). This PoC is built around two docker containers:

\begin{itemize}
\item
  One call streaming that.
\item
  Another one call skills where there where you can find different
  skills for the robots to use. For example Delib (face recognition)
  Affectiva (motion recognition), Brainr (learning by demonstration)
\end{itemize}

(ADD ESQUEMA ROS CLOWD) \#\#\#\#\# ROS Requirements

\hypertarget{requisites}{%
\subsubsection{Requisites}\label{requisites}}

The system needs to be prepared for scalability in two ways:

\begin{itemize}
\item
  The first one it refers the number of robots. We will be able to have
  at least 1 robot at the beginning. But in about a year it is expect to
  have connect in it around one million of robots. So, it is important
  that there is no problem on adding robot very fast.
\item
  The second one it refers on the software side. The system need to be
  able to add functionalities in a easy way. An example of functionality
  may be face detection, Neurolinguistics programming, mapping, a new
  behavior, etc.
\end{itemize}

Another challenge that the system needs to overcome is the fact of being
multiplatform. It needs to be able to communicate with robots of many
brands. On the other way also, it requires integrating behaviors and
functionalities than can be code in multiple languages or run in a
separate different OS.

When you are talking about social robots it is important to mention real
time interactions, that is the reason why the system needs to be able to
have streaming communication.

\hypertarget{propose-of-the-project}{%
\subsubsection{Propose of the project}\label{propose-of-the-project}}

(INSTEAD OF MENTIONING I SHOULD EXPLAIN THEM)

Why SQL or NoSQL? Which can be a good database for saving the logs and
all the knowledge?

Which system, is ROS scalable enough, does it accomplish all our needs?

Which type of Architecture is the best for the cloud system?

\hypertarget{analysis-and-research}{%
\subsection{Analysis and research}\label{analysis-and-research}}

These section is about taking the needs of the project learning the
theory that is related with it and research about which technologies can
be use and which ones may work better.

\hypertarget{big-picture-of-the-system.}{%
\subsubsection{Big picture of the
system.}\label{big-picture-of-the-system.}}

( ESQUEMA SIN HABLAR DE TECNOLOGIAS SOLO CON ELEMENTOS )

\hypertarget{database}{%
\subsubsection{Database}\label{database}}

(BREIEF INTRODUCCION)

\hypertarget{cap-theorem}{%
\subparagraph{CAP Theorem}\label{cap-theorem}}

Also known as Brewer's theorem states that distributed databases cannot
have consistency Availability and Partition Tolerance all at the same
time.

\begin{itemize}
\item
  Consistency: consistent copies of data on different servers.
\item
  Availability: refers to providing a response to any query
\item
  Partition Tolerance: means if a network that connects two or more
  databases and if one servers fails, the servers will still be one
  available with consistent data.
\end{itemize}

\hypertarget{relational-vs-nosql}{%
\paragraph{Relational vs NoSQL}\label{relational-vs-nosql}}

\hypertarget{limitations-of-relational-databases}{%
\subparagraph{Limitations of Relational
Databases}\label{limitations-of-relational-databases}}

Nowadays the amount of data that needs to be stored and the processed
has been increasing in big quantities, and with it new terms have come
such as ``BIG DATA'', ``Internet of Things'' or ``Analytics''. The way
of supporting large numbers of users from companies like Google,
Facebook, Amazon, etc. change the way of managing data than smaller
numbers of business users.

Web applications serving tens of thousands or more users were difficult
to implement with relational databases. Four characteristics of data
management systems that are particularly important for large-scale data
management tasks are:

\begin{itemize}
\item
  Scalability: The way to scale the amount of that in the pass was by
  Scale up (upgraded with more CPUs, additional memory, or faster
  storage devices). Scaling up is always a cost concern, and you always
  are restricted for the hardware capacity. Another option is to use
  multiple servers but managing a relational DB using different servers
  increase the complexity and the difficulty.
\item
  Cost: Upgrading with better hardware always have a huge impact of cost
  ??????
\item
  Flexibility:
\item
  Availability: Being able to querying is one of the big problems when
  you have millions of users accessing the information at almost the
  same time from anywhere in the world. We need to ensure that the
  database has a way to ensure that the data when there are
  infrastructure problems, or system errors. With Relational Databases
  is not an easy problem to solve, it can become a very complicated
  problem when maximizes the availability is needed.
\end{itemize}

\hypertarget{nosql}{%
\subparagraph{NoSQL}\label{nosql}}

The idea of the NoSQL database is to solve the problems that the typical
relational database has. A good definition for a NoSQL is a variety of
different database technologies that are being non-relational as we
mentioned before that is one of the problems that we are having with
very large amount of Data, it is also distributed meaning that they can
be inculcate to multiple processors. Another characteristic of the NoSQL
is that they have to be horizontally scalable, that means that we can
increase the number of the nodes, and not having to scale vertically,
increase the size of the node.

!!DRAW OF VERTICAL AN HORITZONTAL scaling.

When we talk about big data we have 4 dimensions or the 4 V:

\begin{itemize}
\item
  Volume: It refers the amount of data. We have many sources that the
  data can be generated form. For example: Logs, sensors, social media,
  speech, events, etc.
\item
  Variety: It refers the types of data that we can have. It can be
  structured, semi-structured or unstructured. The data can come from
  two main sources: people who create the data and or machines. In the
  case of the people it is going to be more structure. And in the case
  of machines it depends.
\item
  Veracity: How truthy is the data. In some cases, we can have untrusted
  data, for example if you have a node replicated many times and the
  data it is not consistent between them.
\item
  Velocity: The velocity that we need to process the data either the
  speed of generating (and saving) the data and the read of analysis.\\
  https://bicorner.com/2015/06/25/characteristics-that-make-big-data-big/
  (1/12/2017)
\end{itemize}

\hypertarget{types-of-nosql}{%
\subparagraph{Types of NoSQL}\label{types-of-nosql}}

\begin{itemize}
\tightlist
\item
  Column Families:
\end{itemize}

These types of Database are designed to manage big amount of data
distributed on multiple servers in a cluster. If you only have one
server probably these type of data is not what you need.

EXPLAIN HOW IT WORKS

\begin{itemize}
\tightlist
\item
  Document Store
\end{itemize}

A document store allows the inserting, retrieving, and manipulating of
semi-structured data. Most of the databases available under this
category use XML, JSON, with data access typically over HTTP protocol
using RESTful API

Document-oriented databases provide this flexibility---dynamic or
changeable schema or even scheme less documents. Because of the
limitless flexibility provided in this model, this is one of the more
popular models implemented and used. Some of popular databases that
provide document-oriented storage include:

\begin{itemize}
\tightlist
\item
  Key Value
\end{itemize}

When we talk about key value database we need to imagine an array that
have constrains on the keys and the values. That king of array is what
we call associative array. The advantages that you can have on a key
value database over an associative array is the persistence of the data
especially in long term storage.

The idea of a key value is to be able to the fast access of the data,
that is the reason why some key value stores only use memory to keep the
data instead of saving it into the disk.

It is good to implement these kinds of database when you need to use
many times a data, so when we get it form the disk it saves the data to
the cache in order to access it faster.

\begin{itemize}
\tightlist
\item
  Graph Databases
\end{itemize}

Graph Databases as its name suggest, is a database base on the
mathematical element name graph. A graph consists on two parts, vertices
and edges, where a Vertices represents thinks (it can be anything,
cities, animals, stations, etc.) that have relationships with other
thinks. On the other hand, edges are the links that connect vertices and
represents its relationships. The relationships can be long or short
term.

\begin{itemize}
\tightlist
\item
  Multimodal Databases
\end{itemize}

Sometimes one type of database it does not required all the
characteristics of the system. Sometimes is good to combine the
different models for a better performance. For example, when a user logs
into the system it might need to access the data of the user many times,
a good possibility for having better speed is to use a SQL database that
needs to be done

\hypertarget{table-of-competing-technologies}{%
\paragraph{Table of competing
technologies}\label{table-of-competing-technologies}}

\begin{longtable}[]{@{}llcc@{}}
\toprule
\begin{minipage}[b]{0.13\columnwidth}\raggedright
Tecnology\strut
\end{minipage} & \begin{minipage}[b]{0.17\columnwidth}\raggedright
type\strut
\end{minipage} & \begin{minipage}[b]{0.30\columnwidth}\centering
Advantage\strut
\end{minipage} & \begin{minipage}[b]{0.30\columnwidth}\centering
disadvantages\strut
\end{minipage}\tabularnewline
\midrule
\endhead
\begin{minipage}[t]{0.13\columnwidth}\raggedright
Apache Cassandra\strut
\end{minipage} & \begin{minipage}[t]{0.17\columnwidth}\raggedright
column store\strut
\end{minipage} & \begin{minipage}[t]{0.30\columnwidth}\centering
open source. Per to per architecture. Scheme free.\strut
\end{minipage} & \begin{minipage}[t]{0.30\columnwidth}\centering
\$1600\strut
\end{minipage}\tabularnewline
\begin{minipage}[t]{0.13\columnwidth}\raggedright
MongoDB\strut
\end{minipage} & \begin{minipage}[t]{0.17\columnwidth}\raggedright
document store\strut
\end{minipage} & \begin{minipage}[t]{0.30\columnwidth}\centering
Schema less\strut
\end{minipage} & \begin{minipage}[t]{0.30\columnwidth}\centering
\$12\strut
\end{minipage}\tabularnewline
\begin{minipage}[t]{0.13\columnwidth}\raggedright
ArangoDB\strut
\end{minipage} & \begin{minipage}[t]{0.17\columnwidth}\raggedright
Multimodal Database\strut
\end{minipage} & \begin{minipage}[t]{0.30\columnwidth}\centering
w\strut
\end{minipage} & \begin{minipage}[t]{0.30\columnwidth}\centering
\$1\strut
\end{minipage}\tabularnewline
\begin{minipage}[t]{0.13\columnwidth}\raggedright
Hadoop\strut
\end{minipage} & \begin{minipage}[t]{0.17\columnwidth}\raggedright
Colum store\strut
\end{minipage} & \begin{minipage}[t]{0.30\columnwidth}\centering
Open source It can manage multiple nodes acros different servers working
in parallel. It is good to manage large numbers of unstructured data. It
is fault tolerance\strut
\end{minipage} & \begin{minipage}[t]{0.30\columnwidth}\centering
Security: No data encryption. Limitations at the time to improve
efficiency, reliability and integration\strut
\end{minipage}\tabularnewline
\begin{minipage}[t]{0.13\columnwidth}\raggedright
Couch DB\strut
\end{minipage} & \begin{minipage}[t]{0.17\columnwidth}\raggedright
document store\strut
\end{minipage} & \begin{minipage}[t]{0.30\columnwidth}\centering
w\strut
\end{minipage} & \begin{minipage}[t]{0.30\columnwidth}\centering
w\strut
\end{minipage}\tabularnewline
\begin{minipage}[t]{0.13\columnwidth}\raggedright
Elastic\strut
\end{minipage} & \begin{minipage}[t]{0.17\columnwidth}\raggedright
w\strut
\end{minipage} & \begin{minipage}[t]{0.30\columnwidth}\centering
or Azure SQL Database\strut
\end{minipage} & \begin{minipage}[t]{0.30\columnwidth}\centering
w\strut
\end{minipage}\tabularnewline
\bottomrule
\end{longtable}

I will explain the technologies I had use and the reason at ROJECT
DEVELOPMENT section

\hypertarget{architecture}{%
\subsubsection{Architecture}\label{architecture}}

(SHORT INTRODUCCTION)

\hypertarget{soa}{%
\paragraph{SOA}\label{soa}}

\hypertarget{microservices}{%
\paragraph{Microservices}\label{microservices}}

" Self-contained process that provides a unique business capability"
https://www.youtube.com/watch?v=PY9xSykods4

``\ldots{} the microservice architectural style is an approach to
developing a single application as a suite of small services, each
running in its own process and communicating with lightweight
mechanisms, often an HTTP resource API. These services are built around
business capabilities and independently deployable by fully automated
deployment machinery.\\
it is important to focus on single business capabilities.'' {[}The term
``microservice'' was discussed at a workshop of software architects near
Venice in May, 2011 to describe what the participants saw as a common
architectural style that many of them had been recently exploring. In
May 2012, the same group decided on ``microservices'' as the most
appropriate name. James presented some of these ideas as a case study in
March 2012 at 33rd Degree in Krakow in Microservices - Java, the Unix
Way as did Fred George about the same time. Adrian Cockcroft at Netflix,
describing this approach as ``fine grained SOA'' was pioneering the
style at web scale as were many of the others mentioned in this article
- Joe Walnes, Dan North, Evan Botcher and Graham Tackley.{]} x * almost
stateless of the application

IMPORTANT SETTINGS

each group need to have our data model. make communication state less

REAL EXAMPLES: * Amazon

Advantages:

\begin{itemize}
\tightlist
\item
  It is easy to have small teams developing each capability
\item
  Incase that one service broke the others will keep working
\end{itemize}

\hypertarget{acid}{%
\subparagraph{ACID:}\label{acid}}

*Atomicity: A unit that cannot be further divided.

*Consistency: Ensure the integration of data.

*Isolation:

*Durability:

\hypertarget{base}{%
\subparagraph{BASE:}\label{base}}

*Basically:

*Available:

*Soft State:

*Eventually Consistent:

\hypertarget{project-development}{%
\subsection{PROJECT DEVELOPMENT}\label{project-development}}

\hypertarget{introduction-1}{%
\subsubsection{Introduction}\label{introduction-1}}

In this section It describes the research about which technologies that
can be used to solve the database problem, and a comparison between them
to find which ones can be the best. It is also to define the
architecture of the system and which technologies that can be used to
implement it.

\hypertarget{technologies}{%
\subsubsection{Technologies}\label{technologies}}

\hypertarget{tecnologies-for-database}{%
\paragraph{Tecnologies for Database}\label{tecnologies-for-database}}

For crating a database I use Elasandrea, that is a conversion of
Cassandra, an NoSql using the column family model, and Elastic search a
search engine.

\hypertarget{cassandra}{%
\subparagraph{Cassandra:}\label{cassandra}}

Apache Cassandra is open source, distributed data storage system that
differs sharply from relational database management systems.

Cassandra is a NoSQL using Column Store model. It has a primary language
call CQL (Cassandra Query Language) that is use for communicating with
the database.

\begin{figure}
\centering
\includegraphics{./tex2pdf.7640/595570be93c99f9c498ee0b492997683ba024ba3.shtml}
\caption{alt text}
\end{figure}

\hypertarget{the-history-of-cassandra}{%
\subparagraph{The History of Cassandra}\label{the-history-of-cassandra}}

Cassandra first started as an incubation project at Apache in January of
2009. Shortly thereafter, the committers, led by Apache Cassandra
Project Chair Jonathan Ellis, re-leased version 0.3 of Cassandra, and
have steadily made minor releases since that time.

Though as of this writing it has not yet reached a 1.0 release,
Cassandra is being used in production by some of the biggest properties
on the Web, including Facebook, Twitter, Cisco, Rackspace, Digg, Cloud
kick, Reddit, and more. {[}book{]}

Releases after graduation include:

\begin{verbatim}
0.6, released Apr 12 2010, added support for integrated caching, and Apache Hadoop MapReduce 
0.7, released Jan 08 2011, added secondary indexes and online schema changes 
0.8, released Jun 2 2011, added the Cassandra Query Language (CQL), self-tuning memtables, and support for zero-downtime upgrades 
1.0, released Oct 17 2011, added integrated compression, leveled compaction, and improved read performance 
1.1, released Apr 23 2012, added self-tuning caches, row-level isolation, and support for mixed ssd/spinning disk deployments 
1.2, released Jan 2 2013, added clustering across virtual nodes, inter-node communication, atomic batches, and request tracing 
2.0, released Sep 4 2013, added lightweight transactions (based on the Paxos consensus protocol), triggers, improved compactions, CQL paging support, prepared statement support, SELECT column alias support 
\end{verbatim}

\hypertarget{why-cassandra}{%
\subparagraph{Why Cassandra}\label{why-cassandra}}

\begin{itemize}
\item
  Scalability: In the system is very important to design for a
  scalability because the company could decide to add hundreds of new
  robots over a short period of time. So, Cassandra is prepared to have
  a high scalability performance.
\item
  Write Speed: It is important that when the system want to save new
  information it does fast because there may be vast amounts of
  information to save. One of the examples that needs to be fast is when
  you save all the interaction information as a log.
\item
  It provides high availability:
\item
  Designed to manage very large amounts of structured data like logs.
\item
  Peer-to-peer distribution model.
\item
  The gossiper is a protocol communication (to support decentralization
  and partition tolerance) responsible for making sure every node in the
  system eventually knows important information about every other node's
  state, including those that are unreachable or not yet in the cluster
  when any given state change occurs.
\item
  It has Antientropic, it is a synchronization mechanism that consist in
  comparing all the replicas of each piece of data that exist (or are
  supposed to) and updating each replica to the newest version.
\item
  It uses CQL (Cassandra Query Language), it is a subset of SQL and its
  easy for someone coming from SQL to learn.\\
  \#\#\#\#\# Requirements
\item
  minimum 2 cores, recommended 8 or more cores.
\item
  minimum 8GB of RAM, recommended 32GB of RAM.
\item
  Java 8
\item
  Python 2.7
\end{itemize}

\hypertarget{elasticsearch}{%
\paragraph{Elasticsearch}\label{elasticsearch}}

Elasticsearch is a search engine base on RESTful, that have two main
functions: it can be used as an analytic framework and datastore as
well.

It can be used for several cases:\\
* As a Text Search. * Event Data and Metrics. * Visualizing Data. *
Scraping and Combining Public Data. * Logging and Log Analysis.

(HERE I HAVE TO EXPLAIN THAT IS DOCUMENT BASE)

\hypertarget{history}{%
\subparagraph{History}\label{history}}

Elasticsearch had the first version released for Shay Banon in Febrary
2010. Shay Banon had a scalability issue with Compass and he decide to
rewrite and used a common interface, JSON over HTTP, suitable for
programming languages other than Java as well.

\hypertarget{why-elasticsearch}{%
\subparagraph{Why Elasticsearch}\label{why-elasticsearch}}

One of the advantages of Elasticsearch is that you can create a client
made of any language as the answer is a JSON object. The main advantages
of these protocol are:

\begin{itemize}
\item
  Portability: It use web standards so it can be implemented in many
  languages such as Java, Python, Ruby, c\#, etc., or called form
  command lines applications such as curl.
\item
  Durability:
\item
  Simple to use:
\item
  Speed: It can compute complex queries in a short amount of time.
\item
  High support: A lot of plugins uses a REST endpoint on HTTP.
\item
  Scalability: Is the same to talk to Elasticsearch running on a single
  node than hundreds of them. It scales horizontally to handle billions
  of events per second, while automatically managing how indices and
  queries are distributed across the cluster for oh-so smooth
  operations.
\item
  Security: Elasticsearch out of the box it does not have enough
  security control. It use HTTP authentication. But as I mention before
  there are lots of plugins, for example Search Guard is a free plugin
  that you can find role-based access control, document level security
  and SSL/TLS encrypted node-to-node communication. You can also find
  other plugins for security.
\item
  It has a big community that maintain and use it. That can be really
  useful at the time to solve problems or learn how to use it. Another
  reason is because the community apart official rivers that support the
  most use programming languages.
\item
  Resilient, Highly Available
\end{itemize}

\hypertarget{requirements}{%
\subparagraph{Requirements}\label{requirements}}

\begin{itemize}
\tightlist
\item
  Java 8
\item
  curl
\end{itemize}

\hypertarget{elassandra-elasticsearch-cassandra}{%
\paragraph{Elassandra = Elasticsearch +
Cassandra}\label{elassandra-elasticsearch-cassandra}}

``Elassandra is a fork of Elasticsearch modified to run as a plugin for
Apache Cassandra in a scalable and resilient peer-to-peer architecture.
Elasticsearch code is embedded in Cassandra nodes providing advanced
search features on Cassandra tables and Cassandra serve as an
Elasticsearch data and configuration store.''

https://github.com/strapdata/elassandra

\includegraphics{./tex2pdf.7640/0a7e96de27bbfb38b7ecd804c4760fba3f1f5275.shtml}
https://github.com/strapdata/elassandra

\hypertarget{history-1}{%
\subparagraph{History}\label{history-1}}

\hypertarget{characteristics}{%
\subparagraph{Characteristics}\label{characteristics}}

\hypertarget{why-elassandra}{%
\subparagraph{Why Elassandra}\label{why-elassandra}}

\begin{itemize}
\tightlist
\item
  It have an enterprise version, Strapdata
\item
  An Elasticsearch index is mapped to a cassandra keyspace, and a
  document type to a cassandra table.
\item
  Docker
\item
  Durability
\item
  CQL (Common Query Language): it is possible to perform CQL queries to
  .
\item
  It is possible to use the plugins of Elasticsearch.
\item
  Definitely it has the advantages of Elasticsearch and cassandra all
  together.
\end{itemize}

\hypertarget{requirements-1}{%
\subparagraph{Requirements}\label{requirements-1}}

\begin{itemize}
\tightlist
\item
  Java 8
\end{itemize}

\hypertarget{technologies-for-the-architecture}{%
\subsubsection{Technologies for the
architecture}\label{technologies-for-the-architecture}}

\hypertarget{docker}{%
\paragraph{Docker}\label{docker}}

Docker is a container technology that is create \#\#\#\#\# History
\#\#\#\#\# Characteristics \#\#\#\#\# Why Docker

\begin{itemize}
\item
  Portability: Everything required to run the application is packaged
  into a standardized container which means your applications are no
  longer restrained to certain infrastructure requirements. Applications
  can move across multiple environments, development stages, and cloud
  environments - consistently.
\item
  Increase security: The Docker EE platform provides you with all the
  tools and capabilities you need to run containers securely at scale.
  With services like security scanning and container signing, Docker EE
  enables you to protect all app components from the source, across the
  network, and to different collaborators and environments with
  guarantees against tampering.
\item
  Efficiency: By containerizing your legacy application on Docker
  Enterprise Edition (EE), you reduce the total resource requirements to
  run your application. This increases operational efficiency and allows
  you to consolidate your infrastructure.
\item
  Enterprise version (EE) and community version (CE) \#\#\#\#\#
  Requirements
\item
  Available for: Mac, Windows, Debian, Ubuntu, CentOS, Azure and AWS.
\end{itemize}

Depending of the OS. To install Docker CE in Ubuntu, you

\begin{itemize}
\tightlist
\item
  64-bit version o
\item
  Ubuntu versions: Artful 17.10 (Docker CE 17.11 Edge only), Zesty
  17.04, Xenial 16.04 (LTS), Trusty 14.04 (LTS),
\item
  Ubuntu on x86\_64, armhf, and s390x (IBM z Systems) architectures.
\end{itemize}

\hypertarget{kafka}{%
\paragraph{Kafka}\label{kafka}}

\includegraphics{./tex2pdf.7640/21f0ca1de70d48cee774b613334768a53f8a95bc.shtml}
\#\#\#\#\# History\\
\#\#\#\#\# Why kafka \#\#\#\#\# Requirements * Need to have ZooKeeper
install and started. * For ZooKeeper it is need: Supported Platforms *
GNU/Linux. * Sun Solaris. * FreeBSD. * Win32. * MacOSX. JDK 6 or
greater. *

\hypertarget{schedule}{%
\subsubsection{Schedule}\label{schedule}}

A Gantt diagram

\hypertarget{analyzing-ros-ros-architecture.}{%
\subsubsection{Analyzing ros ROS
architecture.}\label{analyzing-ros-ros-architecture.}}

different modules explanation.

\begin{itemize}
\item
  Robot framework: I it is lines of code that are running inside the
  robot and it finality it is for establishing the connection between
  the robot and the server.
\item
  Streaming docker: these block the only things that does is receiving
  the streaming of one camera and save it in a shared memory.
\item
  Skills docker: In these modules there is all the skills and behaviors
  that the robot could have. It reads the shared memory to get the
  images from the robot. It sends what the robot has to do.
\end{itemize}

It only receives one camera at the time.

Every time you want to add a different for the first time: * Create the
robot framework. * Redo some of the skills in order to accept and adapt
the robot.

Every time you want to add a new skill:

\begin{enumerate}
\def\labelenumi{\arabic{enumi}.}
\tightlist
\item
  stop all the system
\item
  Change the docker file to add the new file
\item
  Rebuild the docker.
\item
  Run the streaming Docker.
\item
  Inside the docker run the scrip that receive the streaming.
\item
  Run the skill Doker.
\item
  Inside the docker run the scrip of the skills that you want to use.
\end{enumerate}

As you can see that every time that it is need to add a new skill or
even a little change to one of them, it is need to stop the hold state.

Also, the fact it is not easy to add another camera. In order to do it
is need to run a second streaming Docker and change the skills that are
going to use both cameras.

\hypertarget{solution}{%
\subsection{Solution}\label{solution}}

\hypertarget{architecture-design}{%
\subsubsection{Architecture design}\label{architecture-design}}

In order to accomplish the requirements for this project I decide to use
microservice architecture. It will give to the system some advantage
instead of using the ROS. To implement the architecture, I decide to use
Kafka and Docker.

\hypertarget{kafka-1}{%
\paragraph{Kafka}\label{kafka-1}}

\hypertarget{design-for-scalability}{%
\paragraph{Design for scalability}\label{design-for-scalability}}

\hypertarget{data-structure}{%
\subsubsection{Data structure}\label{data-structure}}

\begin{figure}
\centering
\includegraphics{./tex2pdf.7640/097c69286146b51c5547d5e691c8af617eaab987.shtml}
\caption{alt text}
\end{figure}

\begin{figure}
\centering
\includegraphics{./tex2pdf.7640/67e9638dfef5364f3d32ceccf541f2ce99ec3216.shtml}
\caption{alt text}
\end{figure}

People table:

\begin{verbatim}
{ 
        "pid" : "",       
        "name" : { 
            "first_name" : "", 
            "last_name" : "" 
        }, 
        "personal_info" : { 
            "age" : "", 
            "telephone" : "", 
            "email" : "" 
        }, 
        "last_seen" : { 
        "time_seen" : "", 
        "place_seen" : "" 
    } 
} 
\end{verbatim}

Robot table:

\begin{verbatim}
{ 
    "rId" : "", 
    "rType" : { 
        "name" : "", 
        "version" : "" 
    }, 
    "rOwner" : "" 
} 
\end{verbatim}

Logs interaction table:

\begin{verbatim}
{ 
    "id" : "", 
    "where" : "", 
        "network" : "", 
        "place" : "" 
    }, 
    "when" : { 
        "time" : "", 
        "day" : "" 
    }, 
    "info" : { 
    } 
} 
\end{verbatim}

\hypertarget{functionality}{%
\subsubsection{Functionality}\label{functionality}}

\hypertarget{reflections}{%
\subsection{Reflections}\label{reflections}}

(WHAT ARE THE NEXT STEPS)

\begin{itemize}
\item
  Talk about implementing security
\item
  Talk about adding testing
\item
  Monitoring, my proposals:

  \begin{itemize}
  \tightlist
  \item
    Monitoring elastic search with kibana
  \item
    Monitoring kafka with Cofluents
  \end{itemize}
\item
  Analytics for annualizing the interactions.
\item
\end{itemize}

(What can be improve on the system)

When I design the database, I wanted to find the best way to save the
Logs and the knowledge in one database using one technology. During all
this process I had been improving my understanding of databases and now
I have a better understanding about database design.

(TALK ABOUT HOW AGAIL WAS MY PROJECT)

\hypertarget{conclusions}{%
\subsection{Conclusions}\label{conclusions}}

Microservice architecture. It will give to the system some advantage
instead of using the ROS:

\begin{enumerate}
\def\labelenumi{\arabic{enumi}.}
\tightlist
\item
  There are many technologies out of the box. I chose Kafka, and it is a
  good technology because there are so many things already implemented.
\item
  Better performance of scalability. It is simple fast and easy to add
  new services.
\item
  If one service falls the others ones can keep working.
\end{enumerate}

Benefits of kafka. * API connector for Cassandra and elastyc search and
other databases. The ones I used they are maintained by confluent. * Big
community. It is very helpful at the time to fine divers or solving
problems. * It supports streaming communication. In our system it is
important to have a real time communication as we are working with
robots.\\
* Confluent Control Center it is an out of a box tool for monitoring.\\
* Ready for fault isolation (EXPLAIN MORE AND HOW IT DOES). * It
supports REST.

\hypertarget{section}{%
\subsection{}\label{section}}

\hypertarget{books}{%
\subsubsection{Books}\label{books}}

\begin{verbatim}
author name 
title of the publication (and the title of the article if it's a magazine or encyclopedia) 
date of publication 
the place of publication of a book 
the publishing company of a book 
the volume number of a magazine or printed encyclopedia 
the page number(s) 
\end{verbatim}

NoSQL for Mere Mortals by Dan Sullivan. Published by Addison-Wesley on
April 2015, First printing. Text printed in the United States on
recycled paper at Edwards Brothers Malloy, Ann Arbor, Michigan.

Cassandra: The Definitive Guide, by Eben Hewitt, Published by O'Reilly
Media on November 2010, First Edition.

Kafka: The Definitive Guide, Real-Time Data and Stream Processing at
Scale, by Neha Narkhede, Gwen Shapira, and Todd Palino. Published by
O'Reilly Media on July 2017: First Edition.

The OpenCV Tutorials, Release 2.4.13.3, September 03, 2017 \#\#\# Webs
http://blogs.mindsmapped.com/bigdatahadoop/hadoop-advantages-and-disadvantages/
(1/10/2017)

https://docs.mongodb.com/manual/ (1/10/2017)

https://www.tutorialspoint.com/mongodb/mongodb\_advantages.htm
(20/11/2012)

https://highlyscalable.wordpress.com/2012/03/01/nosql-data-modeling-techniques/
(1/12/2017)

https://www.rosehosting.com/blog/how-to-install-apache-cassandra-on-ubuntu-16-04/
(23/10/2017)

https://docs.datastax.com/en/cql/3.1/cql/cql\_intro\_c.html (17/10/2017)

\hypertarget{elisandra-elasticsearch-cassandra}{%
\subparagraph{Elisandra: Elasticsearch +
Cassandra}\label{elisandra-elasticsearch-cassandra}}

http://doc.elassandra.io/en/latest/installation.html (2/11/2017)

https://github.com/strapdata/elassandra (2/11/2017)

https://www.youtube.com/watch?v=0WuLZTvA3YM (2/11/2017)

\hypertarget{elasticsearch-1}{%
\subparagraph{Elasticsearch}\label{elasticsearch-1}}

https://www.elastic.co/ (17/10/2017)

https://www.3pillarglobal.com/insights/advantages-of-elastic-search
(25/10/2017)
https://apiumhub.com/tech-blog-barcelona/elastic-search-advantages-books/
(11/12/2017)

\hypertarget{cassandra-1}{%
\subparagraph{Cassandra}\label{cassandra-1}}

https://docs.datastax.com/en/cql/3.1/index.html (17/10/2017)

\hypertarget{kafka-2}{%
\subparagraph{Kafka}\label{kafka-2}}

https://kafka.apache.org/documentation/ (10/12/2017)
https://zookeeper.apache.org/doc/r3.1.2/zookeeperAdmin.html\#sc\_systemReq
(11/12/2017)

\hypertarget{appendix}{%
\subsection{Appendix}\label{appendix}}

\hypertarget{basic-concepts}{%
\subsubsection{Basic concepts}\label{basic-concepts}}

Cluster: a cluster can be a group node which can consist of computers,
databases, or other technologies that work together closely

Node: a single computer or database or technology in a system.

Index:

Type:

Document:

Shards and Replicas:

\begin{itemize}
\item
  Container
\item ~
  \hypertarget{abbreviations}{%
  \subsubsection{Abbreviations}\label{abbreviations}}
\end{itemize}

NoSQL - non-relational Structured Query Language

DB - Database

SW - Software

HW - Hardware

OS - Operating system

//think about the reason behind choosing technologies such as
documentation, maturity, and the other 3 things

``Kafka Connect currently supports two modes of execution: standalone
(single process) and distributed.

In standalone mode all work is performed in a single process. This
configuration is simpler to setup and get started with and may be useful
in situations where only one worker makes sense (e.g.~collecting log
files), but it does not benefit from some of the features of Kafka
Connect such as fault tolerance. You can start a standalone process with
the following command:

1 \textgreater{} bin/connect-standalone.sh
config/connect-standalone.properties connector1.properties
{[}connector2.properties \ldots{}{]}

" ¨ ´ " Distributed mode handles automatic balancing of work, allows you
to scale up (or down) dynamically, and offers fault tolerance both in
the active tasks and for configuration and offset commit data. Execution
is very similar to standalone mode:

1 \textgreater{} bin/connect-distributed.sh
config/connect-distributed.properties"

" The difference is in the class which is started and the configuration
parameters which change how the Kafka Connect process decides where to
store configurations, how to assign work, and where to store offsets and
task statues. In the distributed mode, Kafka Connect stores the offsets,
configs and task statuses in Kafka topics. It is recommended to manually
create the topics for offset, configs and statuses in order to achieve
the desired the number of partitions and replication factors. If the
topics are not yet created when starting Kafka Connect, the topics will
be auto created with default number of partitions and replication
factor, which may not be best suited for its usage."

Kafka Connect Elasticsearch:

https://github.com/hannesstockner/kafka-connect-elasticsearch
(1/12/2017)

Prerequisites: * a Linux console or Apple OSX console (not tested on
windows, but adaptable with little effort) * a Git client to fetch the
project * Docker Compose * Apache Maven installed. * git clone
https://github.com/hannesstockner/kafka-connect-elasticsearch
kafka-connect-elasticsearch

Kafka Connect Cassandra:

https://github.com/jcustenborder/kafka-connect-cassandra (1/12/2017)

\end{document}
